\documentclass[english]{article}
\usepackage[utf8x]{inputenc}
\usepackage[T1]{fontenc}
\usepackage{babel}
\usepackage{amsmath}
\usepackage{graphicx}
\usepackage{fancyhdr}
\pagestyle{fancy}
\fancyhf{}
\renewcommand{\headrulewidth}{0pt}
\setlength{\headheight}{40pt} 
\usepackage[a4paper,top=3cm,bottom=2cm,left=3cm,right=3cm,marginparwidth=1.75cm]{geometry}

\begin{document}

\title{\bf Kernel Optimization}

\author{ Enrique Anaya Bovio}

\maketitle
\thispagestyle{fancy}

\vspace{1cm}

{\bf Abstract:} The kernel is a computer program that has complete control over the operating system. It handles the input/output requests and translates them into data-processing instructions for the CPU. It handles not only the memory, but also peripherals. 

The kernel controls the tasks that are managed in the running system, some of these are running processes, hardware management and handling interrupts, in the kernel space. On the other hand, the user performs in the user space. This separation helps to prevent data interfering that can cause instability and slowness, or worse, malfunctioning application programs that can crash the entire operating system. 

The process scheduler decides which is the next task to run. In the following project we analyzed the behavior of the scheduler changing the default value of the runtime scheduling, this value is 950000µs for the scheduler real time running variable. According to this, 5\% of the CPU time is reserved for processes that are not running under a realtime or deadline scheduling policy. On this project, this value specifies how much of the period time could be used by real-time and deadline scheduled processes on the system.


\vspace{1cm}
\begin{center}
{\bf 1. Introduction}
\end{center}

 The kernel controls the tasks that are managed in the running system, in the kernel space, some of these are running processes, hardware management and handling interrupts. The process scheduler is one of the most important components in a multitasking operating system, it is responsible to use the system resources in the best way possible and guarantee the execution of multiple tasks simultaneously.

A typical real-time task is composed by a repetition of computation phases that are activated on a periodic fashion. The usage of a real-time task is defined by the ratio between its WCET and its period, and represents the part of CPU time needed to execute the task.

In the following project we will analyze the behavior of scheduler changing the default value of run time scheduling, in order to understand the quantity of megabits per second depending on the microseconds specified.

\vspace{1cm}
\begin{center}
{\bf 2. Theorical Framework}
\end{center}

{\bf 2.1 The Kernel}

The kernel is the central module of an operating system. It is the part of the operating system that loads first, and it remains in main memory. The kernel code is usually loaded into a protected area of memory to prevent it from being overwritten by programs or other parts of the operating system.

The kernel is responsible for memory management, process and task management, and disk management. The kernel connects the system hardware to the application software. 

\vspace{2.5cm}{\bf 2.2 The Scheduler}

The scheduler is in charge of keeping busy the CPUs in the system. The Linux scheduler implements a number of scheduling policies, which determines how long a thread runs on a particular CPU core and when it runs.

The scheduling policies are divided into two major categories:

\begin{enumerate}
\item Realtime policies:
\begin{itemize}
\item SCHED\_FIFO
\item SCHED\_RR
\end{itemize}

\item Normal policies:
\begin{itemize}
\item SCHED\_OTHER
\item SCHED\_BATCH
\item SCHED\_IDLE
\end{itemize}
\end{enumerate}

For this project we are focused to analyze just the behavior of the realtime policies by changing parameters in /proc/sys/kernel/sched\_rt\_period\_us.

\vspace{0.5cm}{\bf 2.2.1 Realtime scheduling policies}

Normal threads are scheduled after the real-time threads had been scheduled. The realtime policies are used for time-critical tasks that must have complete without any interruption.

{\bf SCHED\_FIFO}

This policy is also referred to as static priority scheduling, because it defines a fixed priority (between 1 and 99) for each thread. The scheduler scans a list of SCHED\_FIFO threads in priority order and schedules the highest priority thread that is ready to run. This thread runs until it blocks, exits, or is preempted by a higher priority thread that is ready to run.

Even the lowest priority realtime thread will be scheduled ahead of any thread with a non-realtime policy; if only one realtime thread exists, the SCHED\_FIFO priority value does not matter.

{\bf SCHED\_RR}

A round-robin variant of the SCHED\_FIFO policy. SCHED\_RR threads are also given a fixed priority between 1 and 99. However, threads with the same priority are scheduled round-robin style within a certain quantum, or time slice. The sched\_rr\_get\_interval(2) system call returns the value of the time slice, but the duration of the time slice cannot be set by a user. This policy is useful if you need multiple thread to run at the same priority


\vspace{0.5cm}{\bf 2.2.1.1 SCHED\_FIFO policy}

In the Linux kernel, the SCHED\_FIFO policy includes a bandwidth cap mechanism. This protects realtime application programmers from realtime tasks that might monopolize the CPU. This mechanism can be adjusted through the following /proc file system parameters:

{\bf /proc/sys/kernel/sched\_rt\_period\_us}

Defines the time period to be considered one hundred percent of CPU bandwidth, in microseconds ('us' being the closest equivalent to 'µs' in plain text). The default value is 1000000µs, or 1 second.


{\bf /proc/sys/kernel/sched\_rt\_runtime\_us}

Defines the time period to be devoted to running realtime threads, in microseconds ('us' being the closest equivalent to 'µs' in plain text). The default value is 950000µs, or 0.95 seconds.

\vspace{4cm}
\begin{center}
{\bf 3. Objectives}
\end{center}

\begin{itemize}
\item Make a kernel change on scheduler/memory managment and measure the performance impact using the phoronix benchmark tools.
\item Analyze and compare the changes in AIO Stress according to the increasement of microseconds.
\item Understand and learn about realtime scheduling.
\end{itemize}

\vspace{1cm}
\begin{center}
{\bf 4. Justification}
\end{center}

The need to learn about how the Linux kernel schedulers works, primarily to make an analysis of the changes to be made in the values of this. Also understand the meaning of the values that we will change and the relationship between the microseconds (run time) and the AIO Stress average.

\vspace{1cm}
\begin{center}
{\bf 5. Development}
\end{center}

The default value of sched\_rt\_runtime\_us is 950000µs (0.95 seconds). According to this, 5\% of the CPU time is reserved for processes that are not running under a realtime or deadline scheduling policy; this value specifies how much of the period time could be used by real-time and deadline scheduled processes on the system. The value can range from -1 (that makes the run-time the same as the period) to INT\_MAX-1, making the run-time the same as the period there's no CPU time set aside for non-realtime processes.

\vspace{0.5cm}{\bf 5.1 System Specifications}
\begin{itemize}
\item Processor: Intel Core i7-4500U @ 3.00GHz (4 cores)
\item Memory: 2 x 4096 MB DDR3-1600MHz
\item Disk: 1000GB Western Digital WD10JPVX-75J
\item Network: Realtek RTL8101/2/6E + Qualcomm Atheros QCA9565 / AR9565
\item Motherboard: Dell 03VVKX
\item Chipset: Intel Haswell-UTL DRAM
\item OS: Ubuntu 16.04
\item Kernel: 4.4.0-43-generic (x86\_64)
\item Compiler: GCC 5.4.0 20160609Motherboard: Dell 03VVKX
\item Display Server: X Server 1.18.4
\item Display Driver: Intel 2.99.917
\item File-System: ext4
\end{itemize}


\begin{figure}
\includegraphics[width=1.0\textwidth]{results_table.png}
\caption{\label{fig:results_table}  Test results by changing the microseconds.}

In Figure 1 the test results are displayed, including standard error and standard deviation as parameter to support the precision of the AIO Stress. The table compares the default value (950,000 microseconds) with the performance of different values that were defined from 100,000 to 1,000,000 microseconds with a difference of 100,000 between each different test.
\end{figure}

\begin{figure}
\includegraphics[width=1.0\textwidth]{results_micro_aio.png}
\caption{\label{fig:results_micro_aio} Relation between microseconds and AIO Stress. }
The Figure 2 shows the results between the relation of Microseconds and AIO Stress. Thanks to the graph, we can see that the AIO Stress varies depending on the values defined. In Figure 3, it is interesting that the values specified that had a lower AIO Stress are the ones that have the higher standard error.
\end{figure}


\begin{figure}
\includegraphics[width=1.0\textwidth]{results_micro_error.png}
\caption{\label{fig:results_micro_error} Relation between microseconds and standard error. }
\end{figure}


\begin{center}
\vspace{28cm}{\bf Conclusion}
\end{center}

	With the realization of this project that was focused on research and analysis of results, it was obtained, among other things, the better understanding of the use and importance of a scheduler within the kernel of a computer. Although this is the essential center of an operating system, which provides the most basic services for the proper functioning, it cannot take all the credit. On the other hand, there is the Shell, which is the most external part of an operating system, which interacts with the commands provided by the user. The role of the scheduler in these two elements of the OS is also very important, it is responsible for keeping busy the CPU of the system, it is a small part of the kernel, but it is the one that coordinates the fluidity of the processes that occur in the system by the user requests through the Shell.

	But with this project we can realize something else, the scheduler, with all the importance it has within the operating system, can be improved because it provides configuration variables that are available so that a user can modify them and achieve an improvement in their OS. In the making of this report we learned to modify these variables to achieve a change in the performance of the computer and, with it, an optimization in it.

	In the end, a lot was learned, but there is more: the developer of an operating system. External to the practice, unconsciously in the end, those people who are dedicated to carry this out are more appreciate. It takes a lot of research and some basic knowledge about the operating system to only modify a variable given by them. Carrying out an entire operating system requires a lot of dedication and an excellent work team. There is much that can be drawn from this report and it is not just the numbers.

\vspace{8cm}
\begin{center}
{\bf References}
\end{center}

Red Hat. (s.f.). CPU Scheduling. Retrieved from: https://access.redhat.com/documentation/enUS/
Red\_Hat\_Enterprise\_Linux/6/html/Performance\_Tuning\_Guide/s-cpu-scheduler.html
    
\vspace{0.3cm}
Silberschatz A, Galvin P, Gagne G. (2012). Operating System Concepts. John Wiley \& Son, Inc. Ninth Edition.

\vspace{0.3cm}
Bladernr. (2017). PhoronixTestSuite. Retrieved from: https://wiki.ubuntu.com/PhoronixTestSuite

\vspace{0.3cm}
Bovet D, Cesati M. (2005). Understanding the Linux Kernel. O'Reilly Media, Inc. Third Edition.


\end{document}